%===================================== ABSTRACT =================================

{\center \section*{Abstract}}


\noindent
Sepsis is one of the leading causes of death for hospitalized patients. Early symptoms are challenging to identify, as they are subtle, and other conditions with a common physiologic response meet the same criteria of diagnosis. During sepsis, the depressed vascular properties of the circulatory system are the main contributor to decreased blood perfusion. The reduced vascular tone will cripple the autoregulatory mechanisms that the body controls to maintain appropriate blood perfusion by controlling pressure and vascular dilation. 

In a healthy patient, the autoregulation is observed as distinct slow variations in the hemodynamic parameters, such as systemic vascular resistance (SVR) and arterial compliance. Provided velocity and pressure measurements, we can calculate the arterial impedance, which can be analogized to a 2-element Windkessel model (WK). The model resembles a parallel RC circuit, in which the resistance is the SVR, and the capacitance is the arterial compliance. During septic shock, the autoregulatory mechanisms will be unstable and show a more stochastic behavior in the 2-element WK model as it attempts to stabilize the blood flow. This stochastic behavior should be derivable as an increase in the magnitude of the low autoregulatory frequencies and its neighbors. 

The 2-element WK model parameters were found from 3-minute blood pressure and velocity measurements of seven patients. The parameters were then transformed into the frequency domain through a digital Fourier transform (DFT), where the possibility of distinguishing the severity of the septic shock from looking at the average magnitude of the working frequencies of the autoregulation was investigated. Although there seems to be some correlation, there is not enough evidence to employ this as a diagnostic measurement for sepsis. I do believe there is potential for further and more comprehensive studies on this matter, and I do see the potential to develop related research with this project's results and potential for development as a baseline.

\cleardoublepage