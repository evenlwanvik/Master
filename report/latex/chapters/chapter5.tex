%===================================== CHAP 4 =================================

\chapter{Discussion}

The theory regarding the autoregulatory mechanisms in a septic patient, in that they are impaired and observable in the hemodynamic parameters, have been proven in the relevant literature. There are abnormalities in both the systemic vascular resistance and the arterial compliance in association with the severity of the septic shock. From figure \ref{fig:resistance_patient18}, one can see that there is a correlation between the resistance and the NE dosage. The compliance in figure \ref{fig:compliance_patient18}, however, seems to show an opposite trend. Most likely, this discrepancy is caused by the vasopressor effects of the NE, as a decrease in the vascular system's ability to dilate will reduce the compliance. The same results seem to be true for the examination of the magnitudes of their frequency components in figure \ref{fig:resistance_dft} and \ref{fig:compliance_dft}. The vasopressor effects should, however, not have too large of an impact on the relative magnitude measures when we divide by the DC component. By transforming the signal to a relative and comparable scale, it should have suppressed the differences in vascular dilation capabilities.

It is evident that the autoregulatory oscillations become unstable and start oscillating stochastically and with higher amplitudes. There is a correlation between the magnitude and the NE dosage, but it is less significant than I expected. It might have been useful to split the frequencies of interest into narrower bands to look for more specific autoregulatory deviances that following the severity of the disease. With more recordings, one could create a method that would crosscheck multiple frequency bands to the NE dosage. 

I do believe there is potential for further and more comprehensive studies on this matter, and I do see the potential to develop related research with this project's results and potential for development as a baseline.



\cleardoublepage