%===================================== CHAP 1 =================================

\chapter{Introduction}

\section{Motivation}
Sepsis is one of the leading causes of death for hospitalized patients. About half the patients who have a septic shock will die within the first month of their diagnosis.\cite{RN2} It is estimated that there will be 18 million cases of sepsis every year \cite{RN1}. Early symptoms are challenging to identify, as they are subtle, and many other conditions with a common physiologic response meet the criteria of diagnosis. Early treatment of sepsis is associated with improved likelihood of recovery. \cite{RN21} Innovations within detecting cardiovascular abnormalities with signal processing would be a good addition to existing biomarkers. 

In theory, there is some correlation between impairments in autoregulation due to its difficulties in controlling the vascular dilation as the immune response tries to fight off the invading pathogens. These impairments are visible in hemodynamic parameters such as systemic vascular resistance (SVR) and arterial compliance. The possibility of looking at the changes in SVR and compliance to detect sepsis, or at least its increased severity, could significantly help improve sepsis diagnostics.

\section{previous work}

There are no relevant literature which has looked at the lower frequencies of the hemodynamic parameters to identify if a patient has sepsis.

Tips her Hans? Droppe seksjonen?


\section{Research aims and objectives}
The project was completed at the Department of Engineering Cybernetics (ITK) in collaboration with Department of Circulation
and Medical Imaging (ISB), NTNU. It is a 7.5 specialization project (TTK4551), where the goal is to specialize in the selected area of research based on scientific methods, collect supplementary information based on literature search and other sources, and combine this with own knowledge into a project report.

The research aim and motivation stems from oral and electronic communication with dr. Daniel Bergum, who works with intensive care medicine at the Faculty of Medicine and Health Science. The main goal of this project is to analyze variations in vascular parameters, such as systemic vascular resistance (SVR) and arterial compliance, as a tool of diagnosis, in patients with sepsis.  


\section{Outline of the report}

The project report is divided into five chapters, excluding the introduction, followed by references. 

The second chapter (after introduction) will introduce the relevant theory and citations needed to comprehend the objectives and methodology of the project. The severity and pathophysiology of sepsis and how it affects the hemodynamic parameters will be explained — followed by the possibility of transforming the parameters into an analogous parallel RC-circuit using a simple lumped 2-element Windkessel (WK) model.

Chapter 3 demonstrates how to estimate the parameters of a 2-element WK model. It operates on the theory that irregularities will occur in the slow varying changes in systemic vascular resistance (SVR) and arterial compliance controlled by autoregulation. The method is summarized by showing how to potentially detect these irregularities by analyzing the parameter's spectral components. The results of the project are then shown in chapter 4 and thereafter discussed in chapter 5.

At the end of the report, chapter 6 concludes the report with a brief summary, with a few of the principal features and details being addressed.


\cleardoublepage